\documentclass[usenames,dvipsnames,10pt,aspectratio=169]{beamer} 

\usepackage[utf8]{inputenc}
\usepackage{verbatim}
\usepackage{minted}
\usepackage{graphicx}
\usepackage{wrapfig}
\usepackage{geometry}
\usepackage{listings}
\usepackage{hyperref}
\usepackage[utf8]{inputenc}
% \usepackage{showframe}
\usepackage{enumitem}
\usepackage{color, xcolor}
\usepackage[document]{ragged2e}
\usetheme{umu}

\usemintedstyle{monokai}

\usepackage{hyperref}
\hypersetup{
    colorlinks=true,
    linkcolor=ucugreyish,
    filecolor=ucured,
    urlcolor=ucublue,
}
\urlstyle{same}

%%% Some useful commands
% pdf-friendly newline in links
\newcommand{\pdfnewline}{\texorpdfstring{\newline}{ }} 
% Fill the vertical space in a slide (to put text at the bottom)
\newcommand{\framefill}{\vskip 0pt plus 1 filll}

%%% Enter additional packages below (or above, I can't stop you)! / Jesper
\renewcommand{\proofname}{\sffamily{Proof}}

% custom fullpage image:
% { % all template changes are local to this group.
%     \setbeamertemplate{navigation symbols}{}
%     \begin{frame}<article:0>[plain]
%         \begin{tikzpicture}[remember picture,overlay]
%             \node[at=(current page.center)] {
%                 \includegraphics[width=\paperwidth,height=\paperheight]{graphics/version-control.png}
%             };
%         \end{tikzpicture}
%      \end{frame}
% }

% custom shell example inplace
% [fragile] frame
% \begin{lstlisting}[language=Bash, style=shellstyle] 
%     username $ echo a
% \end{lstlisting}

% custom code file
% [fragile] frame
% \lstinputlisting[language=Python, style=codestyle]{code/shebang_ex.py}

% presentation template slides usage
% \framecard[color (not working)]{textbuf}
% \framesplit{Header}{picture}{textbuf}
% \framepic{image}{text}
% \lstinputlisting[language=Bash, style=codestyle]{code/namespace_ex.sh}

%%%%%%%%%%%%%%%%%%%%%%%%%%%%%%%%%%%%%%%%%%%%%%%%%%%%%%%%%%%%%%%%%%%%%%%%%%%%%%%%%%%%%
\title{Linux course}
\subtitle{Version control systems}
\date[\today]{\small\today}
\author[Morhunenko Mykola]{Morhunenko Mykola}
\institute{APPS@UCU}

\setlist[itemize, 1]{label=$\color{ucublue} \bullet$, leftmargin=-2mm}

\begin{document}

\begin{frame}
\titlepage
\end{frame}

\begin{frame}{\contentsname}
    \setbeamercolor{background canvas}{bg=ucugrey}
    \tableofcontents
\end{frame}

\section{Version control}
{ % all template changes are local to this group.
    \setbeamertemplate{navigation symbols}{}
    \begin{frame}<article:0>[plain]
        \begin{tikzpicture}[remember picture,overlay]
            \node[at=(current page.center)] {
                \includegraphics[width=\paperwidth,height=\paperheight]{graphics/version-control.png}
            };
        \end{tikzpicture}
     \end{frame}
}

\begin{frame}{Version control systems}    
    \begin{itemize}
        \item Sooner or later, during the development process, it is necessary to check, 
        what was before, how it became broken
        \item Maybe it's easier to use \ex{Ctrl+z}, but it's impossible to check what was three weeks 
        ago with any keybinding
        \item So in 1972 people started to think about \ex{version control systems}
        \item Firstly, it had been just a tool for saving a history of binary files, but in 1977 
        the first \ex{source code control system} was introduced
        \item The main idea behind - to save the program source code on some checkpoints
        \ex{(commits)}, add features, develop them leaving trunk untouchable\ex{(branches)}, \ex{merge} new features with a trunk, and release some \ex{tags}
        \item Since than, the concept itself was developed, and a lot of version control systems have apeared
    \end{itemize}
\end{frame}

\section{git}
{ % all template changes are local to this group.
    \setbeamertemplate{navigation symbols}{}
    \begin{frame}<article:0>[plain]
        \begin{tikzpicture}[remember picture,overlay]
            \node[at=(current page.center)] {
                \includegraphics[keepaspectratio,width=\paperwidth]{graphics/git.jpg}
            };
        \end{tikzpicture}
     \end{frame}
}

\begin{frame}{Linus again...}
    \begin{itemize}
        \item \ex{Linux kernel} is a huge project, and it is important to have some source-control management system (SCMs) to maintain it
        \item From 2002 to 2005 BitKeeper, a proprietary SCMs was used to maintain the project
        \item At some point (3 April 2005), Linus Torvalds realized, that existing tools are not suitable for Linux development, so in three days he anounced a project and become a self-hosting of \ex{Git} on the next day
        \item It was totally different SCMs. Linus maintained it for half a year, and Junio Hamano has been the core maintainer since then
        \item It was open-source, free software, with a very strong safeguards against corruption, either accidental or malicious
        \item Torvalds sarcastically quipped about the name \ex{Git}, means \ex[ucured]{unpleasant person} in British English
        \item He said: \ex[ucuyellow]{\textit{"I'm an egotistical bastard, and I name all my projects after myself. First 'Linux', now 'git'."}} =)
    \end{itemize}
\end{frame}

\framesplit{What makes Git so good}{graphics/distrib.png}{
    \begin{itemize}
        \item Strong support for non-linear development
        \item Distributed development
        \item Efficient handling of large projects
        \item Toolkit-based design
        \item Pluggable merge strategies
        \item And more other features
        \item It's hard to find any statistics, but that is clear - Git is the most popular SCMs of ourdays
    \end{itemize}
}

\section{GitHub}
{ % all template changes are local to this group.
    \setbeamertemplate{navigation symbols}{}
    \begin{frame}<article:0>[plain]
        \begin{tikzpicture}[remember picture,overlay]
            \node[at=(current page.center)] {
                \includegraphics[keepaspectratio,width=\paperwidth]{graphics/github.jpg}
            };
        \end{tikzpicture}
     \end{frame}
}

\framesplitc{Git is not GitHub}{graphics/gitvshub.png}{
    \begin{itemize}
        \item \ex{GitHub, Inc} - provifer of internet hosting for software development and version control using git
        \item It offers all the functionality of \ex{Git} + it`s own features
        \item Since 2018 - subsidiary of \ex[ucured]{Microsoft}
        \item \ex[ucured]{Not an Open Source project}, but there is a forum for feature requests...
        \item As of January 2020, GitHub reports having over 40 million users
        \item More about it's features after\ex{Git usage} part
    \end{itemize}
}


\section{GitLab}
{ % all template changes are local to this group.
    \setbeamertemplate{navigation symbols}{}
    \begin{frame}<article:0>[plain]
        \begin{tikzpicture}[remember picture,overlay]
            \node[at=(current page.center)] {
                \includegraphics[keepaspectratio,width=\paperwidth]{graphics/gitlab.png}
            };
        \end{tikzpicture}
     \end{frame}
}
\framesplit{Git is not GitHub}{graphics/gitlab-vs-github.jpg}{
    \begin{itemize}
        \item First difference - GitLab was created by \ex{Ukrainian} people, in Ukraine =)
        \item It has deffinitely more features, than\ex{GitHub}, but there is no critical difference
        \item It is \ex[ucuyellow]{Open Source}, unlike github
        \item It is possible to have the same repositoru on both servers, and I do it sometimes
        \item So as for 2021 it's just a question of taste
    \end{itemize}
}

\section{Git usage}
\framecard{GIT USAGE}
{ % all template changes are local to this group.
    \setbeamertemplate{navigation symbols}{}
    \begin{frame}<article:0>[plain]
        \begin{tikzpicture}[remember picture,overlay]
            \node[at=(current page.center)] {
                \includegraphics[keepaspectratio,width=\paperwidth]{graphics/gitmeme.jpg}
            };
        \end{tikzpicture}
     \end{frame}
}

\begin{frame}[fragile]{Creating a repo}
    \begin{itemize}
        \item A \ex{repository} contains all of your project's files and each file's revision history. You can discuss and manage your project's work within the repository.
        \item \ex{Repository} is NOT a project folder. Repository is s a \textit{data structure that stores metadata for a set of files or directory structure}
        \item Command to inicialize a repo in your current folder
        \begin{lstlisting}[language=Bash, style=shellstyle] 
git init
\end{lstlisting}
        \item use \ex{git add} command to cpecify files you want to track, followed by \ex{git commit} - add a cpecifick message to your commit
        \begin{lstlisting}[language=Bash, style=shellstyle] 
git add *.sh
git add .gitignore
git commit -m "add gitignore file; add scripts for some task" \end{lstlisting}
        \item How to write correct commit messages is another art, but remember to write meaningdull messages
        \item So at this point, we have a Git repository in our project directory with tracked files and an initial commit
    \end{itemize}
\end{frame}

\framesplitc{Changes to the repo}{graphics/lifecycle.png}{
    \begin{itemize}
        \item At this point you have a git\ex{repo}with scripts and some files
        \item Each file in your working directory can be in one of two states: tracked or untracked
        \item Tracked files are files that were in the last snapshot, as well as any newly staged files; they can be unmodified, modified, or staged 
        \item In short, tracked files are files that Git knows about
        \item Untracked files are everything else
        \item Use\ex{git status}to check the status of each file in current directory
        \item Files in a\ex{.gitignore}are ignored by git repo
    \end{itemize}
}


\begin{frame}[fragile]{Manipulations with repo}
    \begin{itemize}
        \item As far as\ex{git}is a\ex{decenralized}system, you already have your repo with all version control features
        \item But now about the most powerful\ex{git}feature and why we use it -\ex{remote repo}
        \item You can either\ex{clone}existing repo, or add a\ex{remote}to local one
        \item 
    \end{itemize}
\end{frame}

\section{GitHub features}


\section{Sources}
\framecard{Sources}
\begin{frame}{Sources}
    \begin{itemize}
       \item \href{https://en.wikipedia.org/wiki/Comparison_of_version-control_software}{Version control systems comparison}
       \item \href{https://translatedby.com/you/why-git-is-better-than-x/original/}{Why Git is Better than X}
       \item \href{https://en.wikipedia.org/wiki/Git}{Git Wiki}
       \item \href{https://en.wikipedia.org/wiki/GitHub}{GitHub Wiki}
       \item \href{https://about.gitlab.com/company/history/}{GitLab history}
       \item \href{https://docs.github.com/en}{GitHub documentation}
       \item \href{https://git-scm.com/}{Git documentation}
    \end{itemize}
\end{frame}

\end{document}
