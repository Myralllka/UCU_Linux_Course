\documentclass[usenames,dvipsnames,10pt,aspectratio=169]{beamer} 

\usepackage[utf8]{inputenc}
\usepackage{verbatim}
\usepackage{minted}
\usepackage{graphicx}
\usepackage{wrapfig}
\usepackage{geometry}
\usepackage{listings}
% \usepackage{showframe}
\usepackage{enumitem}
\usepackage{color, xcolor}
\usepackage[document]{ragged2e}
\usetheme{umu}

\usemintedstyle{monokai}

\usepackage{hyperref}
\hypersetup{
    colorlinks=true,
    linkcolor=ucugreyish,
    filecolor=ucured,
    urlcolor=ucublue,
}
\urlstyle{same}

%%% Some useful commands
% pdf-friendly newline in links
\newcommand{\pdfnewline}{\texorpdfstring{\newline}{ }} 
% Fill the vertical space in a slide (to put text at the bottom)
\newcommand{\framefill}{\vskip 0pt plus 1 filll}

%%% Enter additional packages below (or above, I can't stop you)! / Jesper
\renewcommand{\proofname}{\sffamily{Proof}}

% custom fullpage image:
% { % all template changes are local to this group.
%     \setbeamertemplate{navigation symbols}{}
%     \begin{frame}<article:0>[plain]
%         \begin{tikzpicture}[remember picture,overlay]
%             \node[at=(current page.center)] {
%                 \includegraphics[width=\paperwidth,height=\paperheight]{graphics/version-control.png}
%             };
%         \end{tikzpicture}
%      \end{frame}
% }

% custom shell example inplace
% [fragile] frame
% \begin{lstlisting}[language=Bash, style=shellstyle] 
%     username $ echo a
% \end{lstlisting}

% custom code file
% [fragile] frame
% \lstinputlisting[language=Python, style=codestyle]{code/shebang_ex.py}

% presentation template slides usage
% \framecard[color (not working)]{textbuf}
% \framesplit{Header}{picture}{textbuf}
% \framepic{image}{text}
% \lstinputlisting[language=Bash, style=codestyle]{code/namespace_ex.sh}

%%%%%%%%%%%%%%%%%%%%%%%%%%%%%%%%%%%%%%%%%%%%%%%%%%%%%%%%%%%%%%%%%%%%%%%%%%%%%%%%%%%%%
\title{Linux course}
\subtitle{Linux File systems}
\date[\today]{\small\today}
\author[Morhunenko Mykola]{Morhunenko Mykola}
\institute{APPS@UCU}

\setlist[itemize, 1]{label=$\color{ucublue} \bullet$, leftmargin=-2mm}

\begin{document}

\begin{frame}
\titlepage
\end{frame}

\begin{frame}{\contentsname}
    \setbeamercolor{background canvas}{bg=ucugrey}
    \tableofcontents
\end{frame}

\begin{frame}{Intro}
    \begin{itemize}
        \item This is not an overview of some \ex{hardware} memory staff
        \item Neither a pesentation with deep File systems implementation details
        \item More about that you should learn at the \ex{Operating systems} course
        \item This is just an overview of \ex{file systems} that system administraitors use in their everyday life
        \item If you think that you are not a system administrator - think one more time, because you administrate your own system every day
    \end{itemize}
\end{frame}

\section{File system for users}
\framepic{graphics/membrain.jpg}{\hspace{-0.5cm}Memory}

\framesplit{Drives}{graphics/ssdhdd.jpeg}{
    \begin{itemize}
        \item All data stored on some physical devices
        \item It has different storage approaches on each device (HDD, SSD, CD, DVD, Flash, RAM, DDR memory modules)
        \item But now we are going to overview the memory from\ex{user point of view}
        \item How to manage files and file systems, how to chose the most suitable
    \end{itemize}
}

\framesplitc{Memory storage}{graphics/memory.png}{
    \begin{itemize}
        \item Memory as abstraction looks like an array, where bites are stored one by one in a row
        \item\ex{File system}- a method of data structure that the operating system uses to control how data is stored and retrieved
        \item A\ex{file}is an ordered collection of data blocks
        \item In Linux system, everything is a file and if it is not a file, it is a process
        \item So File systems are very important for this OS
    \end{itemize}
}

\section{Everything is a file}
{ % all template changes are local to this group.
    \setbeamertemplate{navigation symbols}{}
    \begin{frame}<article:0>[plain]
        \begin{tikzpicture}[remember picture,overlay]
            \node[at=(current page.center)] {
                \includegraphics[width=\paperwidth,height=\paperheight]{graphics/file.jpg}
            };
        \end{tikzpicture}
        \Huge\textbf{Everything is a file}
     \end{frame}
}

\framesplit{File types}{graphics/types.png}{
    \begin{itemize}
        \item There are a lot of file types, but the most important for us are:
        \item \ex{Regular Files} - some files with data stored inside
        \item \ex{Directories} - files, that allowed to group other files and keep tree filesystem structure
        \item \ex{Character files} - for simulating character devices as terminals, keyboard, network etc
        \item \ex{Block files} - for modelling block devices as disks, flash drives
        \item \ex{Links} - entry points to other files
        \item There are\ex{pipes}, \ex{sockets}
    \end{itemize}
}

\framesplitc{File metadata}{graphics/metha.jpg}{
    \begin{itemize}
        \item File also save a \ex{metadata} about itself, as:
        \item Protection, password
        \item Creator, owner
        \item Flags (r w x)
        \item Size 
        \item Creation time, last update time (timestamp)
    \end{itemize}
}

\section{Fyle systems types}

\begin{frame}{Fyle systems types}

    

\end{frame}

\section{Working with file systems}

\section{Mounting}

\section{Linux Filesystems Hierarchy (LFS)}

\begin{frame}{Linux Filesystems Hierarchy (LFS)}
    \begin{itemize}
        \item This topic worth a separate lecture, but lets make it short 
        \item 
    \end{itemize}
\end{frame}


\section{Sources}
\framecard{Sources}
\begin{frame}{Sources}
    \begin{itemize}
        \item UCU Linux Club
        \item \href{https://en.wikipedia.org/wiki/File_system}{File systems Wiki}
        \item \href{https://www.javatpoint.com/linux-files}{Linux file systems}
    \end{itemize}
\end{frame}

\end{document}