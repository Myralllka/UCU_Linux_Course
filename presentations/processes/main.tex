\documentclass[usenames,dvipsnames,10pt,aspectratio=169]{beamer} 

\usepackage[utf8]{inputenc}
\usepackage{verbatim}
\usepackage{minted}
\usepackage{graphicx}
\usepackage{wrapfig}
\usepackage{geometry}
\usepackage{listings}
% \usepackage{showframe}
\usepackage{enumitem}
\usepackage{color, xcolor}
\usepackage[document]{ragged2e}
\usetheme{umu}

\usemintedstyle{monokai}

\usepackage{hyperref}
\hypersetup{
    colorlinks=true,
    linkcolor=ucugreyish,
    filecolor=ucured,
    urlcolor=ucublue,
}
\urlstyle{same}

\addtobeamertemplate{navigation symbols}{}{%
    \usebeamerfont{footline}%
    \usebeamercolor[fg]{footline}%
    \hspace{1em}%
    \insertframenumber/\inserttotalframenumber
}
%%% Some useful commands
% pdf-friendly newline in links
\newcommand{\pdfnewline}{\texorpdfstring{\newline}{ }} 
% Fill the vertical space in a slide (to put text at the bottom)
\newcommand{\framefill}{\vskip 0pt plus 1 filll}

%%% Enter additional packages below (or above, I can't stop you)! / Jesper
\renewcommand{\proofname}{\sffamily{Proof}}

% custom fullpage image:
% { % all template changes are local to this group.
%     \setbeamertemplate{navigation symbols}{}
%     \begin{frame}<article:0>[plain]
%         \begin{tikzpicture}[remember picture,overlay]
%             \node[at=(current page.center)] {
%                 \includegraphics[width=\paperwidth,height=\paperheight]{graphics/version-control.png}
%             };
%         \end{tikzpicture}
%      \end{frame}
% }

% custom shell example inplace
% [fragile] frame
% \begin{lstlisting}[language=Bash, style=shellstyle] 
%     username $ echo a
% \end{lstlisting}

% custom code file
% [fragile] frame
% \lstinputlisting[language=Python, style=codestyle]{code/shebang_ex.py}

% presentation template slides usage
% \framecard[color (not working)]{textbuf}
% \framesplit{Header}{picture}{textbuf}
% \framepic{image}{text}
% \lstinputlisting[language=Bash, style=codestyle]{code/namespace_ex.sh}

%%%%%%%%%%%%%%%%%%%%%%%%%%%%%%%%%%%%%%%%%%%%%%%%%%%%%%%%%%%%%%%%%%%%%%%%%%%%%%%%%%%%%
\title{Linux course}
\subtitle{Provesses}
\date[\today]{\small\today}
\author[Morhunenko Mykola]{Morhunenko Mykola}
\institute{APPS@UCU}

\setlist[itemize, 1]{label=$\color{ucublue} \bullet$, leftmargin=-2mm}

\begin{document}

\begin{frame}[noframenumbering]
\titlepage
\end{frame}

\begin{frame}{\contentsname}
    \setbeamercolor{background canvas}{bg=ucugrey}
    \tableofcontents
\end{frame}

\section{Processes. Introduction}
\framecard{Processes. Introduction}
\begin{frame}
    \frametitle{Processes}
    \begin{itemize}
        \item In our case -\ex{process}- an instance of a running program with it's resources
        \item But what is the difference between the program and the process?
        \item \ex{Program}- a file, containing some information that describes how to construct a process in a runtime
        \item It includes:
        \item Binary format identification - some metainformation about the format of executable file. Nowedays UNIX executable files called\ex{Executable and Linking format (ELF)}
        \item Machine-language instructions - main algorithm of the program
        \item Program entry-point address
        \item Data
        \item Symbol and relocation tables
        \item Some other information, more about that on the\ex[ucublue]{Operating systems course}
        \item But what is the process? Long story, let's begin
    \end{itemize}
\end{frame}

\section{Processes explanation}

\begin{frame}[fragile]
    \frametitle{Process. PID}
    \begin{itemize}
        \item The very first thing, that is assosiated with any process, it's\ex{PID} - process id 
        \item It's a positive integer, and system works with processes by their PID's - names (commands) are for humans 
        \item There are no fixed ID's for any process, with exception of\ex{init}(more about that in the next topic). PID for\ex{init}equals 1
        \item Maximum PID number for your OS can be found using the following command:
        \begin{lstlisting}[language=Bash, style=shellstyle] 
username $ cat /proc/sys/kernel/pid_max\end{lstlisting}
        \item Also one more important PID for all processes - parrent PID or PPID
        \item If parrent of any process "died" - the child become "adopted" by the\ex{init}process
        \item Parent of any process can be found like:
        \begin{lstlisting}[language=Bash, style=shellstyle] 
username $ cat /proc/PID/status | grep PPid \end{lstlisting}
    \end{itemize}
\end{frame}

\begin{frame}
    \frametitle{Page table}
    \begin{itemize}
        \item 
    \end{itemize}
\end{frame}

\section{Scedualing}

\section{Real time processes}

\section{Sources}
\framecard{Sources}
\begin{frame}{Sources}
    \begin{itemize}
        \item \href{https://tldp.org/LDP/tlk/kernel/processes.html}{Linux processes}
        \item "Linux programming interfaces", M. Kerrisk
    \end{itemize}
\end{frame}

\end{document}